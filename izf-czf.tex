\documentclass[12pt, a4paper]{article}
%\usepackage[utf8]{inputenc}
\usepackage{amsmath,amssymb,amsthm}
\usepackage{extsizes}
\usepackage{tcolorbox}
\usepackage[english]{babel}
%\usepackage{algorithm}
\usepackage{braket}
\usepackage[noend]{algpseudocode}
%\frenchspacing % No double spacing between sentences
%\linespread{1.2} % Set linespace
\usepackage[lmargin=0.1\paperwidth, rmargin=0.1\paperwidth, tmargin=0.1\paperheight, bmargin=0.1\paperheight]{geometry} %margins
%\usepackage{parskip}
\usepackage{titletoc}
\usepackage{mathptmx}
\usepackage{bbold}
\usepackage{graphicx}
\title{{\bf IZF/CZF} Handout}
\author{Raja Damanik\footnote{Master of Logic student, University of Amsterdam. Email: rajaoktovin@gmail.com}}
\date{}

\theoremstyle{definition}
\newtheorem{definition}{Definition}
\usepackage{bm}

\theoremstyle{plain}
\newtheorem{theorem}{Theorem}
\newtheorem{lemma}{Lemma}


\newcommand{\enc}[1]{\langle #1 \rangle}
\newcommand{\dis}[1]{\lVert #1 \rVert}
\DeclareMathOperator*{\argmin}{arg\,min}
\DeclareMathOperator*{\argmax}{arg\,max}
\begin{document}
\maketitle

\section{Motivation}

In similar spirit of classical set theory, intuitionistic set theory {\bf IZF} and constructive set theory {\bf CZF} are meant to provide the most powerful possible foundations or formalism for mathematicians (or logicians) who work in intuitionistic or constructive mathematics.
The necessity of doing so is caused by the fact that some axioms in classical logic imply some laws that cannot be accepted in intuitionistic logic.
Consequently, it is instructive to get rid of axioms that are not compatible with intuitionistic logic.
Furthermore, if we want to be interested in foundation of constructive mathematics, it is also desirable to have an axiom that comes with good justification for why it is constructive.

There are three kinds of axioms in classical set theory on its nature...


We use intuitionistic logic instead of classical logic.
Another important thing to consider is the way the axiom is written.
This is because two axioms may be equivalent in classical logic but not equivalent in intuitionistic logic. 
For example, there are two equivalent ways of writing foundation axiom, one stating that there is a minimal element in every set and the other one stating there is no infinite descending $\in$-chain, but they are not equivalent in intuitionistic logic.

In this note, I will write a brief introduction to intuitionistic logic in Section 2. In Section 3, I give a table that puts the axioms in the theories side by side. The main content is Section 4, where I discuss each axiom shown in Section 3 and what it facilitates us in doing mathematics, the philosophical motivation of adding or removing them from the theory, and how it relates to other axioms.

\section{Intuitionistic logic}

A brief section that shows how to use intuitionistic logic and its power relative to classical logic.


\section{Bird eye's view}

We will talk about some important axioms in detail below.
This table gives a bird eye's view of which axioms are used in which theory.

We also introduce ${\bf CZF}_0$ as a fragment of ${\bf CZF}$ in order to talk about equivalence of some axioms in ${\bf CZF}$.
\begin{center}
\begin{tabular}{ |c|c|c|c| } 
\hline
${\bf ZF}$ & ${\bf IZF}$ & ${\bf CZF}_0$ & ${\bf CZF}$ \\ \hline 
Extensionality & Extensionality & Extensionality & Extensionality\\ 
Pairing & Pairing & Pairing & Pairing \\
Union & Union & Union & Union \\
Powerset & Powerset & - & \underline{\emph{Subset Collection}} \\
Infinity & Infinity & \emph{Strong Infinity} &  \emph{Strong Infinity} \\
Foundation & \emph{Set Induction} & - & \emph{Set Induction} \\
Separation & Separation & \emph{Bounded Separation} & \emph{Bounded Separation} \\
Replacement & \emph{Collection} & Replacement & \underline{\emph{Strong Collection}} \\
\hline
\end{tabular}
\end{center}

\section{The axioms}
\subsection{Extensionality, Pairing, and, Union}

There are three axioms in classical set theory that are considered non-controversial in {\bf IZF} and {\bf CZF}.
\begin{itemize}
\item {\bf Extensionality:} 
\begin{align*}
(\forall x)(\forall y)(x=y \leftrightarrow (\forall z)(z \in x \leftrightarrow z \in y)).
\end{align*}
Axiom of extensionality talks about the equality of sets: they are equal if and only if they contain exactly the same elements.
This axiom is not controversial in {\bf IZF} and {\bf CZF}.

\item {\bf Pairing:} 
\begin{align*}
(\forall x)(\forall y)(\exists z)(\forall u)(u \in z \leftrightarrow u=x \vee u=y).
\end{align*}
Axiom of pairing talks about the existence of set of the form $\{x,y\}$; together with axiom of extensionality, such set is unique.
This set is also considered harmless in {\bf IZF} and {\bf CZF} due to its relatively simple nature.

\item {\bf Union:} 
\begin{align*}
(\forall x)((\exists y)(y \in x)\rightarrow (\exists y)(\forall z)(z \in y \leftrightarrow (\exists u)(z \in u \in x))) 
\end{align*}
With union axiom, we can construct set $\bigcup x = \{y:(\exists z)(y \in z \in x)\}$ from a non-empty set $x$; again together with axiom of extensionality, such set is unique.
We can think that such construction is harmless since every element $y$ of $\bigcup x$ has an already-available witness for it to be included namely the element of $z$ in our given $x$.
\end{itemize}

\subsection{Separation and Bounded Separation}

We consider separation axiom and its fragment called bounded separation axiom.
Separation axiom allows us to define a new set that is a subset of a set $x$ whose elements are specified by some property $\phi$.
This axiom is very handy and is still compatible with intuitionistic logic and hence is still part of {\bf IZF}.

In constructive set theory, more requirement is desired.
In this theory, the predicativity matters more and hence one should only define a set whose requirement is binding on previously constructed set only.
To formalize this, a formula $\phi$ is called $\Delta_0$-formula, if every quantifier that appears in $\phi$ is bounded (namely, of the form $(\forall x \in y)$ or $(\exists x \in y)$.
Then bounded separation axiom (sometimes also called $\Delta_0$-separation or restricted separation) is similar axiom with separation but the formula that separates is only restricted to $\Delta_0$.
\begin{itemize}
\item {\bf Separation:} If $\phi$ is a formula where $y$ does not occur freely in $\phi$, then there exists a set
\begin{align*}
(\forall x)(\exists y)(\forall z)(z \in y \leftrightarrow z \in x \wedge \phi(z)).
\end{align*}
\item {\bf Bounded Separation:} If $\phi$ is a $\Delta_0$-formula where $y$ does not occur freely in $\phi$, then 
\begin{align*}
(\forall x)(\exists y)(\forall z)(z \in y \leftrightarrow z \in x \wedge \phi(z)).
\end{align*}
\end{itemize}

On side note, general comprehension principle (in place of separation axiom) still gives us inconsistent theory intuitionistically:
\begin{itemize}
\item {\bf General comprehension:} For every formula $\phi$ where $x$ does not occur freely in $\phi$:
\begin{align*}
(\exists x)(\forall y)(y \in x \leftrightarrow \phi(y)).
\end{align*}
\end{itemize}
\begin{theorem}
(Russell's Paradox)
General comprehension principle is not valid.
\end{theorem}
\begin{proof}
By general comprehension principle, $Y=\{x|x \notin x\}$ is a set.

Firstly, assume $Y \in Y$. 
By definition of $Y$, we have $Y \notin Y$.
Together with our assumption $Y \in Y$, we reach contradiction.
Hence, we have proved that $Y \notin Y$.
But since $Y \notin Y$, then $Y \in Y$ by definition of $Y$ again.
Thus, general comprehension principle gives us contradiction, i.e. general comprehension is not valid.
\end{proof}

\subsection{Replacement, Collection, and Strong Collection}

% A formula $\phi(x,y)$ is called \emph{function formula} if for every $x$ there exists a unique $y$ such that $\phi(x,y)$.
% If $\phi(x,y)$ is a function formula and $z$ is a set we write $F_\phi[z]=\{y:(\exists x)(x \in z \wedge \phi(x,y)\}$, namely the image of $F$ from $z$.

We now look at the following three axioms:
\begin{itemize}
\item {\bf Replacement:} For any formula $\phi(x,y)$ where $z$ does not occur freely in $\phi$:
\begin{align*}
(\forall a)((\forall x \in a)(\exists ! y)\phi(x,y) \rightarrow (\exists b)(\forall x)(x \in b \leftrightarrow (\exists y \in a)\phi(x,y))).
\end{align*}

The replacement axiom tells that if the domain of a function is a set, then the image is also a set. 
\item {\bf Collection:} For any formula $\phi(x,y)$ where $z$ does not occur freely in $\phi$:
\begin{align*}
(\forall a)((\forall x \in a)(\exists  y)\phi(x,y) \rightarrow (\exists b)(\forall x \in a)(\exists y \in b)\phi(x,y)).
\end{align*}

In natural language, the axiom of collection for formula $\phi$ tells us that given a set $a$ and a total relation with domain $a$, then there exists a set $b$ that collects at least one image of each element of $a$.

\item {\bf Strong collection:} For any formula $\phi(x,y)$ where does not occur freely in $\phi$:
\begin{align*}
(\forall a)(((\forall x \in a)(\exists y)\phi(x,y)) \rightarrow (\exists b)((\forall y) 
\end{align*}
\end{itemize}

The philosophical discussion about the intuitionistic or constructive version of axiom of replacement is a bit more complex.
Some papers discussing this are for example in ... \textcolor{red}{To do: add reference}

\subsection{Infinity and Strong Infinity}

Now we look at infinity and strong infinity axiom.
We give an abbreviation of formula as follows $\text{succ}(x,y):=(\forall z)(z \in y \leftrightarrow z \in x \vee z=x)$ and $\text{ind}(a):= \emptyset \in a \wedge (\forall y)(y \in a \rightarrow (\exists z \in a)\text{succ}(y,z))$.
Here, $\emptyset$ is an abbreviation for the unique set (by extensionality) that contains no element.

\begin{itemize}
\item {\bf Infinity:} $(\exists a)(\text{ind}(a))$.
\item {\bf Strong Infinity:} $(\exists a)(\text{ind}(a) \wedge (\forall b)(\text{ind}(b) \rightarrow (\forall x \in a)(x \in b))).$
\end{itemize}

Infinity axiom tells us that there exists a inductive set, namely the set such that for every element of it, there exists another element which contain that set and all of its element.
With infinity axiom in ${\bf ZF}$, one can show define the set of all natural numbers as the least inductive set.
This argument uses full separation axiom.
Due to its impredicativity, we do not use full separation axiom and instead bounded separation takes place in ${\bf CZF}$.
Since least inductive set is a nice way to talk about the set of natural numbers, in ${\bf CZF}$, we strengthen the infinity axiom to strong infinity axiom that states there exists an inductive set and the least inductive set.

\subsection{Foundation and Set Induction}

\begin{itemize}
\item {\bf Foundation:} $$(\forall x)(\exists y) (y \in x \wedge x \cap y=\emptyset)$$

In classical logic, there are two ways of stating foundation axiom (or also called regularity axiom).
One way of doing it is to say that every element has a minimal element, which is as presented above.
Another way of saying it is by the fact that there exists no descending $\in$-chain.
\item {\bf Set Induction:} For all formula $\phi$: $$(\forall a)((\forall x \in a)\phi(x) \rightarrow \phi(a)) \rightarrow (\forall a)\phi(a).$$

As we will show below, axiom of foundation as presented above is too strong for intuitionistic logic as it implies {\bf LEM}. 
On the other hand, if another way of presenting it in classical logic using infinite descending $\in$-chain is used, we are unable to prove some property that is shared by all sets.
One can show that foundation axiom indeed implies set induction; so we add a slightly weaker axiom that still allows us to do something considered important in mathematics.
\end{itemize}

Aside from axiom of foundation, we would like to also look at Foundation Schema:
\begin{itemize}
\item {\bf Foundation Schema:} For any formula $\phi$: $$(\exists x)\phi(x) \rightarrow (\exists x)(\phi(x) \wedge (\forall y \in x)\neg \phi(y)).$$
\end{itemize}

We now see how adding these axioms 

\begin{theorem}
\begin{enumerate}
\item[(i)] ${\bf CZF}+\text{Foundation Schema}={\bf ZF}$.
\item[(ii)] ${\bf CZF}+\text{Separation+Foundation Schema}={\bf ZF}$.
\item[(iii)] ${\bf CZF}$+\text{Foundation Axiom} $\vDash {\bf REM}$.
\item[(iv)] ${\bf CZF}$+\text{Foundation Axiom}$\vDash$Power Set.
\end{enumerate}
\end{theorem}
\begin{proof}
\textcolor{red}{to be completed}
\end{proof}

As corollary, in {\bf IZF}, axiom of foundation implies bounded version of {\bf LEM}, since bounded separation takes place instead of full separation.

\subsection{Powerset and Subset Collection}

One of the most interesting discussions in  the axiomatic freedom in constructive set theory is the power set axiom.
\begin{itemize}
\item {\bf Powerset:}
$$(\forall x)(\exists y)(\forall z)(z \in y \leftrightarrow (\forall u)(u \in z \rightarrow u \in x)).$$

Powerset axiom states that given a set $X$, the powerset $\mathcal{P}(X)$ exists. 

\item {\bf Subset Collection:} For every formula $\psi(x,y,u)$, we have
\begin{align*}(\forall a)(\forall b)((\exists c)(\forall u)((\forall x \in a)(\exists y \in b)\psi(x,y,u) \rightarrow \\ 
(\exists d \in c)((\forall x \in a)(\exists y \in d)\psi(x,y,u) \wedge (\forall y \in d)(\exists x \in a) \psi(x,y,u)))).
\end{align*}

\item {\bf Fullness:} Given sets $A,B$, define $\text{mv}(^A B)$ be the class of all sets $R \subseteq A \times B$ such that $(\forall u \in A)(\exists v \in B) \langle u,v \rangle \in R\}$. A set $C$ is \emph{full in} $\text{mv}(^A B)$ if $C \subseteq \text{mv}(^A B)$ and $(\forall R \in \text{mv}(^A B))(\exists S \in C)S \subseteq R$. For all sets $A,B$, there is a set $C$ that is full in $\text{mv}(^A B)$.


\item {\bf Exponentiation:} If $A$ and $B$ are sets, then $^A B=\{f|f:A \rightarrow B\}$ is also a set.
\end{itemize}

The following theorem gives an idea of what subset collection can do and how it is stated in a language that is more familiar; using total relation.
\begin{theorem}
\begin{enumerate}
\item (${\bf CZF}_0$) Subset collection implies Exponentiation.
\item (${\bf CZF}_0$ + Strong Collection) Subset collection is equivalent to Fullness.
\end{enumerate}
\end{theorem}

\begin{proof}
\textcolor{red}{to be completed}
\end{proof}


\section{Axiom of Choice}

We state below axiom of choice and some versions of axiom of choice.
\begin{itemize}
\item {\bf Axiom of Choice (AC):}
\item {\bf Axiom of Dependent Choice (DC):}
\item {\bf Axiom of Choice - Omega (${\bf AC_\omega}$):}
\end{itemize}

\section{Conclusion}

In this note, we have discussed some axioms that are used in intuitionistic set theory ${\bf IZF}$ and constructive set theory ${\bf CZF}$.
In constructive set theory, it seems like the each axiom requires more philosophical justification to be included in the theory; hence seems weaker than ${\bf CZF}$.



\end{document}
